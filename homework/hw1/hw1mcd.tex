\documentclass[12pt,letterpaper]{article}
\usepackage{amsmath,amsthm,amsfonts,amssymb,amscd}
\usepackage{fullpage}
\usepackage{lastpage}
\usepackage{enumerate}
\usepackage{fancyhdr}
\usepackage{mathrsfs}
\usepackage{xcolor}
\usepackage[margin=3cm]{geometry}
\setlength{\parindent}{0.0in}
\setlength{\parskip}{0.05in}

\usepackage{../compsci430}


\newenvironment{answer}[1]{
  \subsubsection*{Question #1}
}


% \pagestyle{fancyplain}
% \headheight 35pt
% \lhead{\yourname\ \login\\\course\ --- \semester}
% \chead{\textbf{\Large Homework \hwnum}}
% \rhead{\hwdate}
\headsep 10pt

\begin{document}

\cps590header

% \noindent \emph{Homework Notes:} Add information here on your study group, number of hours you spent on the homework, and other relevant information.

\begin{answer}{1}
From largest to smallest, the asymptotic size of the given functions are (with $lg$ taken to represent $log_2$ and $lg^{1000}$ taken to mean $(lg(n))^{1000}$): 
\begin{eqnarray}
&& lg(n^{1000}) \\ 
&& (lg n)^n \\
&& n^{lg n} \\
&& n^{lg lg n} \\
&& (lg n)^{(lg n)} \\
&& lg^{1000}n \\
&& n^2 \\
&& n lg n \\
&& n \\
&& lg n \\
&& n^{(1/lgn)} \\
&& (1+0.001)^n \\
&& lg_{1000} n \\
&& 1
\end{eqnarray}

\end{answer}

\begin{answer}{2} 
The answers to this question rely on the discussion of the master method in chapter 4 of the CLRS textbook. 

\paragraph{2.1)} $T(n) = 2 T(n/3) + 1$, so $a=2, b=3, f(n)=1)$. Using the master method, this is case 1: $f(n) = O(n^{\log_3 2 - \epsilon})$, so $\mathbf{T(n)=\Theta(n^{\log_3 2})}$. 

\paragraph{2.2)} $T(n) = 5 T(n/4) + n$, so $a=5, b=4, f(n)=n$. Again this is case 1 of the master method: $f(n)=O(n^{\log_4 5 - \epsilon})$, so $\mathbf{T(n)= \Theta (n^{\log_5 4 })}$. 

\paragraph{2.3)} $T(n) = 8 T(n/2) + n^3$, so $a=8, b=2, f(n)=n^3$. This is case 2 of the master method: $f(n)= \Theta (n^{\log_2 8})$, so $\mathbf{ T(n)= \Theta (n^3 lg n) }$

\paragraph{2.4)} $T(n) = T(n^{1/2})+1$. This recurrence relation cannot be directly translated into the form of the master method, so we use a change of variables $m=lg n$. $S(m)=S(m/2)+1$, so $a_m=1, b_m=2, f(m)=1$. This is case 2 of the master method, because $S(m)=\Theta(m^{\log_2 1})=\Theta(m lg m)$. Changing back to our original variables, $T(n)=T(2^m)=S(m)=\Theta(m lg m)=\mathbf{ \Theta(lg n~lg lg n)} $

\end{answer}

\begin{answer}{3}
\end{answer}

\begin{answer}{4}
\end{answer}


\end{document}
