\documentclass[12pt,letterpaper]{article}
\usepackage{amsmath,amsthm,amsfonts,amssymb,amscd}
\usepackage{fullpage}
\usepackage{lastpage}
\usepackage{enumerate}
\usepackage{fancyhdr}
\usepackage{mathrsfs}
\usepackage{xcolor}
\usepackage[margin=3cm]{geometry}
\setlength{\parindent}{0.0in}
\setlength{\parskip}{0.05in}

\usepackage{../compsci430}


\newenvironment{answer}[1]{
  \subsubsection*{Question #1}
}


\long\def\cps590header{\begin{center}
\large\bf CPS 430/590.06 \hfill Prof.\ John Reif\\
\large\bf Design and Analysis of Algorithms \hfill Fall 2013 \\
\large\bf Homework 1 \hfill Matt Dickenson
\end{center}}

\headsep 10pt

\begin{document}

\cps590header

\begin{answer}{1}
From largest to smallest, the asymptotic size of the given functions are (with $lg$ taken to represent $log_2$ and $lg^{1000}$ taken to mean performing the $lg$ operation one-thousand times): 
\begin{eqnarray}
&& lg(n^{1000}) \\ 
&& (lg n)^n \\
&& n^{lg n} \\
&& n^{lg lg n} \\
&& (lg n)^{(lg n)} \\
&& n^2 \\
&& n lg n \\
&& n \\
&& lg n \\
&& n^{(1/lgn)} \\
&& (1+0.001)^n \\
&& lg_{1000} n \\
&& lg^{1000}n \\
&& 1
\end{eqnarray}

\end{answer}

\begin{answer}{2} 
The answers to this question rely on the discussion of the master method in chapter 4 of the CLRS textbook. 

\paragraph{2.1)} $T(n) = 2 T(n/3) + 1$, so $a=2, b=3, f(n)=1)$. Using the master method, this is case 1: $f(n) = O(n^{\log_3 2 - \epsilon})$, so $\mathbf{T(n)=\Theta(n^{\log_3 2})}$. 

\paragraph{2.2)} $T(n) = 5 T(n/4) + n$, so $a=5, b=4, f(n)=n$. Again this is case 1 of the master method: $f(n)=O(n^{\log_4 5 - \epsilon})$, so $\mathbf{T(n)= \Theta (n^{\log_5 4 })}$. 

\paragraph{2.3)} $T(n) = 8 T(n/2) + n^3$, so $a=8, b=2, f(n)=n^3$. This is case 2 of the master method: $f(n)= \Theta (n^{\log_2 8})$, so $\mathbf{ T(n)= \Theta (n^3 lg n) }$

\paragraph{2.4)} $T(n) = T(n^{1/2})+1$. This recurrence relation cannot be directly translated into the form of the master method, so we use a change of variables $m=lg n$. $S(m)=S(m/2)+1$, so $a_m=1, b_m=2, f(m)=1$. This is case 2 of the master method, because $S(m)=\Theta(m^{\log_2 1})=\Theta(m lg m)$. Changing back to our original variables, $T(n)=T(2^m)=S(m)=\Theta(m lg m)=\mathbf{ \Theta(lg n~lg lg n)} $

\end{answer}

\begin{answer}{3}
\end{answer}

\begin{answer}{4}
Suppose we label the three pegs $P_1, P_2$, and $P_3$. Without loss of generality, we can assume that initially all $n$ pegs are on $P_1$ and our goal is to move them to $P_3$ under the constraints posed in the question. For convenience we also label the $n \geq 1$ discs $d_1, \ldots, d_n$ where $d_1$ is the smallest and $d_n$ is the largest disc. Let $T(n)$ denote the number of steps required for $n$ discs.

Our algorithm must first handle the base case, which is trivial: when $n=1$, move $d_n$ directly from $P_1$ to $P_3$. For any input $n$, this is the final step. 

In general, we can solve the the Towers of Hanoi problem in three steps with recursion:
\begin{enumerate}
\item Move all discs $d_1, \ldots,d_{n-1}$ from $P_1$ to $P_2$ 
\item Move disc $d_n$ from $P_1$ to $P_3$
\item Move discs $d_1, \ldots, d_{n-1}$ from $P_2$ to $P_3$. 
\end{enumerate} 

Notice that by relabeling the discs that step 1 is equivalent to solving the problem at level $T(n-1)$ if our original goal was to move from $P_1$ to $P_2$. Similarly, step 3 is equivalent to $T(n-1)$ if the original goal was to move from $P_2$ to $P_3$. Step 2 will always take exactly 1 operation. This gives us the recurrence relation $\mathbf{T(n)=2T(n-1) + 1}$. 

We can get the total number of moves as a function of $n$ using this fact. To help understand the recurrence relation it is helpful to examine a few cases with small $n$. We can easily see that in the base case $T(n=1)=1$. Thus, $T(n=2)=2(1)+1=3$. Further, 
\begin{eqnarray*}
T(n=3) &=& 2(3)+1  \\
&=& 2(2(1)) + 2(1) + 1 \\
&=& 2^2+2^1+2^0 \\
&=& 7.
\end{eqnarray*}

More generally, $T(n)$ can be viewed as a summation series: $T(n)=\sum_{i=0}^{n-1} 2^i=2^n -1$. To see this, we use the recurrence relation $T(n)=2 T(n-1) + 1$ and see that at each step we add $2^{n-1}$ steps (this is also apparent from the result in Method 1 above). In the base case we already have a minimum of $2^1 - 1=1$ steps. For $n=2$ we add $2^{2-1}=2$ steps for a total of $2+1=3$ steps. In general, at the $n^{th}$ step we are adding $2^{n-1}$ steps to a total of $2^{n-1}-1$ steps from the $(n-1)^{th}$ iteration. Because $2^{n-1}+2^{n-1}=2\times2^{n-1}=2^n$, we can simplify the series into the closed form equation $2^n - 1$ steps. 

That is, in general for input of size $n$ the Towers of Hanoi algorithm above requires $\mathbf{2^n-1}$ \textbf{operations}.
\end{answer}


\end{document}
